\documentclass{article}
\usepackage[utf8]{inputenc}
\usepackage[T1]{fontenc}
\usepackage{lmodern}
\usepackage{graphicx}
\usepackage[french]{babel}
\usepackage[bookmarks=true]{hyperref}
\usepackage{geometry}
\geometry{hmargin=2cm,vmargin=2cm}

\title{Mémo Vim}
\author{Adrian Poiget}

\begin{document}
\maketitle
\begin{abstract}
Ce document contient une liste non exhaustive des différents raccourcis sur VIM.
J'espère qu'elle pourra vous être utile.\\
Les raccourcis seront représenté comme dans le tableau suivant~:
\begin{tabular}{|p{3cm}| l| }
\hline
Raccourcis & Action\\ \hline
\end{tabular}\\
Pour les raccourcis, la casse est importante : les lettres en majuscule sont l'équivalent de la combinaison Maj-lettre. Par exemple, V correspond à Maj-v et v correspond uniquement à la touche v.
\end{abstract}

\tableofcontents
\newpage

\section{Se déplacer}
\subsection{Déplacement de base}
\begin{tabular}{|p{3cm}| l| }
\hline
h & Aller à gauche \\ \hline
j & Aller en bas \\ \hline
k & Aller en haut \\ \hline
l & Aller à droite \\ \hline
\end{tabular}
ou alors on peut aussi utiliser les flèches.

\subsection{Déplacement par mots}
\begin{tabular}{|p{3cm}| l| }
\hline
w & Aller au début du mot suivant\\ \hline
W & Aller au début du MOT suivant\\ \hline
e & Aller à la fin du mot suivant\\ \hline
E & Aller à la fin du MOT suivant\\ \hline
b & Aller au début du mot précédent\\ \hline
B & Aller au début du MOT précédent\\ \hline
    ge & Aller à la fin du mot précédent\\ \hline
gE & Aller à la fin du MOT précédent\\ \hline
\end{tabular}\\

mot vs MOT : mot est une chaine de lettres, chiffres et underscores, MOT est une chaine de caractères sans espace\\

\subsection{Déplacement par ligne}
\begin{tabular}{|p{3cm}| l| }
\hline
0 (zero) & Aller au début de la ligne \\ \hline
\$ & Aller à la fin de la ligne \\ \hline
\^~(accent circonflexe) & Aller au premier caratère non vide de la ligne \\ \hline
g\_ & Aller au dernier caratère non vide de la ligne \\ \hline
+ & Aller au premier caractère non blanc de la ligne suivante \\ \hline
- & Aller au premier caractère non blanc de la ligne précédente \\ \hline
"Entrée" & Aller au premier caractère non blanc de la ligne suivante\\ \hline
:0 & Aller au début du fichier (ligne 0) \\ \hline
gg & Aller au début du fichier (ligne 0) \\ \hline
1G & Aller au début du fichier (ligne 0) \\ \hline
:\$ & Aller à la fin du fichier\\ \hline
G & Aller à la fin du fichier\\ \hline
CTRL-d & Aller à la fin du fichier\\ \hline
:n & Aller à la ligne n\\ \hline
nG & Aller à la ligne n\\ \hline
ngg & Aller à la ligne n\\ \hline
\end{tabular}\\[0.5cm]


\subsection{Déplacement dans le fichier}
\begin{tabular}{|p{3cm}| l| }
\hline
  50\% & Aller à 50\% du fichier.(milieu du fichier) \\ \hline
  100l & Aller au 100ème caractere depuis la position actuelle \\ \hline
  100 (100 suivi d'un espace) & Aller au 100ème caractere depuis la position actuelle \\ \hline
  25| & Aller au 25ème caractère de la ligne en cours\\ \hline
  :goto 25 & Aller au 25ème caractère depuis le début du fichier\\ \hline
\end{tabular}\\

\subsection{Déplacement par occurence}
\begin{tabular}{|p{3cm}| l| }
\hline
* & Aller à la prochaine occurence du mot sous le curseur \\ \hline
\# & Aller à la précédente occurence du mot sous le curseur \\ \hline
fx & Déplace le curseur à la prochaine occurence du caractère x \\ \hline
Fx & Déplace le curseur à la précédente occurence du caractère x \\ \hline
tx & Déplace le curseur juste avant la prochaine occurence du caractère x \\ \hline
Tx & Déplace le curseur juste avant la précédente occurence du caractère x \\ \hline
\end{tabular}\\

Pour ce déplacer à l'occurence suivante, il suffit d'appuyer sur ; et pour aller à l'occurrence précédente c'est le raccourcis ,
    
\begin{tabular}{l l}
Exemple&	4fx recherche la 4ième occurrence de x.\\
&dt" supprime tout jusqu'au prochain ".\\
&2dfx supprime tout jusqu'à la 2eme prochaine occurence de x.\\
\end{tabular}\\


\subsection{Déplacement par écran}
\begin{tabular}{|p{3cm}| l| }
\hline
CTRL-n & Descendre le curseur d'une ligne \\ \hline
CTRL-p & Monter le curseur d'une ligne\\ \hline
CTRL-f & Descendre le curseur d'une page\\ \hline
CTRL-b & Monter le curseur d'une page\\ \hline
CTRL-d & Descendre le curseur d'une demi-page\\ \hline
CTRL-u & Monter le curseur d'une demi-page\\ \hline
CTRL-e & Descendre l'affichage d'une ligne\\ \hline
CTRL-y & Monter l'affichage d'une ligne\\ \hline
L & Aller au bas de l'écran\\ \hline
xL & Placer le curseur à x lignes du bas de l'écran \\ \hline
H & Aller en haut de l'écran \\ \hline
xH & Placer le curseur à x lignes du haut de l'écran \\ \hline
M & Aller au milieu de l'écran\\ \hline
\end{tabular}\\

z<Enter> or zt Redraw the screen with the current line under the cursor at the top of the screen.\\
z.  or zz Redraw the screen with the current line under the cursor at the middle of the screen.\\
z- or zb Redraw the screen with the current line under the cursor at the bottom of the screen.\\

z<Enter> (screen line on the top)\\
88z<Enter> positions line 88 at the top.\\
z- (scrolls line to the end of the screen)\\
z. (Center of the screen)

\subsection{Déplacement sur une ligne de "plusieurs lignes" visuellement }
\begin{tabular}{|p{3cm}| l| }
\hline
gj & Descendre d'une "ligne"\\ \hline
gk & Monter d'une "ligne"\\ \hline
g0 & Aller au début de la "ligne"\\ \hline
gm & Aller au milieu de la "ligne"\\ \hline
g\$ & Aller à la fin de la "ligne"\\ \hline
g\^~ & Aller au premier caratère non vide de la ligne\\ \hline
\end{tabular}\\

\subsection{Déplacement dans du texte}
\begin{tabular}{|p{3cm}| l| }\hline
\{ & Aller au début du paragraphe \\ \hline
\} & Aller au début du paragraphe suivant \\ \hline
( & Aller au début de la phrase \\ \hline
) & Aller à la phrase suivant \\ \hline
\end{tabular}\\

\subsection{Déplacement dans du code}
\begin{tabular}{|p{3cm}| l| }
\hline
\% & Aller à la parenthèse, accolade, crochet correspondante : \{\}, [], (), or <> \\ \hline
[i & sur une variable : Affiche la ligne contenant la définition de la variable  \\ \hline
[d & sur une macro :  Affiche la définition de la macro \\ \hline
[I & sur une variable : Affiche toutes les lignes contenant la variable \\ \hline
gd & Aller à la définition de la variable locale (1ère occurrence de la variable dans la fonction) \\ \hline
gD & Aller à la définition de la variable locale (1ère occurrence de la variable du fichier)  \\ \hline
[( & Aller à précédente ( \\ \hline
[) & Aller à prochaine ) \\ \hline
[\{ & Aller à précédente \{ \\ \hline
[\} & Aller à prochaine \} \\ \hline
[[ & Aller au précédent bloc "primaire" defini par des accolades   \\ \hline
]] & Aller au prochain bloc "primaire" defini par des accolade \\ \hline
[<CTRL-i> & Aller à la première correspondance (1ère ligne du fichier contenant le terme) \\ \hline
]<CTRL-i>& Aller à la prochaine correspondance (1ère ligne du fichier après le curseur contenant le terme)\\ \hline
\end{tabular}\\

CTRL-i est Tab, donc, [<Tab> va aussi à la première correspondance. 
v\% Select between braces
    
\subsection{Sauts}
Lorsquen vous vous déplacer dans le fichier grâce aux raccourcis précédents, vous pouvez revenir à la position d'avant avec les raccourcis suivavnts~: \\


\begin{tabular}{|p{3cm}| l| }
\hline
CTRL-o & Revenir à l'emplacement précédent\\ \hline
CTRL-i ou tab & Aller à l'emplacement suivant\\ \hline
5CTRL-o & Sauter au 5ème emplacement précédent\\ \hline
5CTRL-i & Sauter au 5ème emplacement suivant\\ \hline
:jumps & Afficher les postions de jumps\\ \hline
\end{tabular}\\

\section{Rechercher}
\begin{tabular}{|p{3cm}| l| }
\hline
/chaine & Recherche le mot vers le bas\\ \hline
?chaine & Recherche le mot vers le haut\\ \hline
\end{tabular}\\

\noindent
Aller à la prochaine occurence de la chaine dans la même direction en tapant n.\\
Aller à la prochaine occurence de la chaine dans la même direction en tapant N

Les expressions regulières peuvent être utiliser pour rechercher
. (dot)	un caractère sauf "aller à la ligne"
*	zero ou plusieurs occurences 
[...]	n'importe quel caractère de l'ensemble
[\^...]	n'importe quel caractère qui n'est pas dans l'ensemble
\^	Anchor - début de la ligne
\$	Anchor - fin de la ligne
<	Anchor - début du mot
>	Anchor - fin du mot

/jo[ha]n 	Search john or joan
/\textbackslash{}< the 	Search a word beginning by " the"
/the\textbackslash{}> 		Search a word ending by "the"
/\textbackslash{}< the\textbackslash{}> 	Search the
/\textbackslash{}< Š.\textbackslash{}> 	Search all words of 4 letters ???
/\textbackslash{}/ 	Search fred but not alfred or frederick ????
/fred\textbackslash{}|joe 	Search fred or joe
%/\<\d\d\d\d\> 	Search exactly 4 digits
%/^\n\{3} 	Find 3 empty lines
:bufdo /searchstr/ 	Search in all open files


:set ic 		pour ignorer la casse
:set hlsearch	pour surligner le mot correspondant
:set incsearch	pour commencer a chercher avant d'voir taper entrée
:set hls is
 :nohlsearch
 
 
\section{Information}
Voici maintenant quelques commandes pour obtenir des informations sur la ligne ou le caractère\\


\begin{tabular}{|p{3cm}| l| }
\hline
:= & Affiche le nombre de ligne\\ \hline
CTRL-g & Affiche des informations sur le ligne en cours\\ \hline
g CTRL-g & Affiche des informations détaillées sur le ligne en cours\\ \hline
:.= & Affiche le numero de la ligne en cours\\ \hline
:\# & Affiche le numero de la ligne en cours\\ \hline
:num & Affiche le numero de la ligne en cours\\ \hline
:/foo/= & Affiche le numero de la prochaine ligne contenant foo\\ \hline
ga & Affiche la valeur hex et ascii du caractère\\ \hline
g8 & Affiche la valeur hex du caractère\\ \hline
\end{tabular}\\

\section{Insertion/Edition}
\subsection{Insertion}
\begin{tabular}{|p{3cm}| l| }
\hline
i & Passer en mode insertion avant le curseur\\ \hline
I & Passer en mode insertion au début de la ligne\\ \hline
a & Passer en mode insertion après le curseur\\ \hline
A & Passer en mode insertion à la fin de la ligne\\ \hline
o & Ajouter une ligne en dessous et passer en mode insertion \\ \hline
O & Ajouter une ligne au dessus et passer en mode insertion \\ \hline
\end{tabular}\\
\subsection{Edition}
\begin{tabular}{|p{3cm}| l| }
\hline
r & Remplacer un caractère (5ra remplace les 5 caractère par a)\\ \hline
R & Remplacer plusieurs caractères\\ \hline
cw & Effacer le mot et passer en mode insertion (changement)\\ \hline
cc ou S & Effacer la ligne entière et passer en mode insertion (changement)\\ \hline
c\$ ou C & Effacer jusqu'à la fin de la ligne et passer en mode insertion (changement)\\ \hline
cl ou s & Effacer le caractère et passer en mode insertion \\ \hline
xp & transposer deux caractères\\ \hline
J & joindre 2 lignes (nJ pour joindre n lignes)\\ \hline
gJ &joindre 2 lignes sans l'espace\\ \hline
\end{tabular}\\

select all lines ggVG or ggHG (gggHG ?)
gf go to the file named by the word
u undo
U undo on the current line
Ctrl-r redo
    
\subsection{Suppression (couper)}
\begin{tabular}{|p{3cm}| l| }\hline
x ou dl & supprimer le caractère sous le curseur (4x en supprime 4)\\ \hline
X ou dh & supprimer le caractère placé avant le curseur\\ \hline
dd ou :.d & supprimer la ligne entière (3dd poursupprimer 3 lignes)\\ \hline
dw & supprimer la fin du mot (d4w delete 4 words)\\ \hline
de & supprimer la fin du mot mais pas l'espace de fin \\ \hline
db & supprimer le début du mot mais pas l'espace de début\\ \hline
dge & supprimer le début du mot et l'espace de début\\ \hline
d\$ or D & supprimer jusqu'à la fin de la ligne\\ \hline
d0 & supprimer le début de la ligne\\ \hline
diw & supprimer le mot (excluant l'espace de fin)\\ \hline
daw & supprimer le mot (incluant l'espace de fin)\\ \hline
dG & supprimer la fin du fichier\\ \hline
dgg & supprimer le début du fichier\\ \hline
:\$d & Supprimer la dernière ligne \\ \hline
:'ad & Suppriemr la ligne avec le marqueur a \\ \hline
:n,md & Supprimer les lignes n à m \\ \hline
:\%d & Supprimer toutes les lignes  \\ \hline
\end{tabular}\\


d2l supprimer 2 caractères à la droite du curseur\\
d4\} supprimer 4 paragraphes\\

\subsection{Copier}
Pareil que pour les commandes pour supprimer en remplaçant d par y\\


\begin{tabular}{|p{3cm}| l| }\hline
yl & Copier le caractère sous le curseur \\ \hline
yh & Copier le caractère à gauche du curseur \\ \hline
yy , :y ou :.y ou Y & Copier la ligne \\ \hline
nyy ou :ny & Copier n ligne \\ \hline
:\$y & Copier la dernière ligne \\ \hline
:'ay & Copier la ligne avec le marquer a \\ \hline
:n,my & Copier les lignes n à m \\ \hline
:\%y & Copier toutes les lignes \\ \hline
yw & Copier la fin du mot \\ \hline
ye & Copier la fin du mot mais sans l'espace de fin\\ \hline
yb & Copier le debut du mot\\ \hline
yge & Copier le debut du mot avec l'espace de début\\ \hline
y\$ & Copier la fin de la ligne \\ \hline
y0 & Copier le début de la ligne \\ \hline
yiw & Copier le mot (excluant l'espace de fin)\\ \hline
yaw & Copier le mot (incluant l'espace de fin)\\ \hline
yG & Copier la fin du fichier\\ \hline
ygg & Copier le début du fichier\\ \hline
y4\} & Copier 4 paragraphes\\ \hline
"+y & Copier la selection dans le presse papier du système\\ \hline
"ay & Copier la selection dans le registre a\\ \hline
\end{tabular}\\
y3h : copy three characters to the left


\subsection{Coller}
\begin{tabular}{|p{4cm}| l| }\hline
\multicolumn{2}{|l|}{\textbf{Mode normal }} \\ \hline
p & Coller après le curseur \\ \hline
P & Coller avant le curseur \\ \hline
"xp ou "xP &  Coller le registre x avant/après le curseur \\ \hline
\multicolumn{2}{|l|}{\textbf{Mode insertion}} \\ \hline
CTRL-R a & Coller le registre a \\ \hline
CTRL-R + ou CTRL-R * & Coller le presse papier du système\\ \hline
\end{tabular}\\
CTRL-R=5*5    : insert 25 into text


\subsection{Changer la casse}
\begin{tabular}{|p{3cm}| l| }\hline
    $\sim$ & Changer la casse d'une lettre \\ \hline
guu ou Vu& Toute la ligne est mise en minuscule\\ \hline
gUU ou VU & Toute la ligne est mise en majuscule\\ \hline
g$\sim$$\sim$ or g$\sim$g$\sim$ & Inverser la casse de la ligne\\ \hline
vEU & Mettre le mot en majuscule \\ \hline
vEu & Mettre le mot en minuscule\\ \hline
vE$\sim$ & Inverser la casse du mot\\ \hline
ggguG & Tout le fichier est mis en minuscule\\ \hline
gggUG & Tout le fichier est mis en majuscule\\ \hline
ggg$\sim$G & Inversse la casse de tout le fichier \\ \hline
<selection>U & La sélection est mise en majuscule) \\ \hline
<selection>u & La sélection est mise en minuscule \\ \hline
g$\sim$motion & (It does not depend on tildeop) \\ \hline
gUmotion & (All uppercase) \\ \hline
gumotion & (All uppercase) \\ \hline
gUw & Mettre le mot en majuscule (de même avec W pour le MOT)\\ \hline
guw & Mettre le mot en minuscule (de même avec W pour le MOT)\\ \hline
g$\sim$w & Inverser la casse du mot \\ \hline
    :\%s/\textbackslash{}\textless{}./\textbackslash{}u\&/g & Mettre la 1ère lettre de chaque mot en majuscule \\ \hline
    :\%s/\textbackslash{}\textless{}./\textbackslash{}l\&/g & Mettre la 1ère lettre de chaque mot en minuscule \\ \hline
    :\%s/.*/\textbackslash{}u\& &	Mettre la 1ère lettre de chaque ligne en majuscule\\ \hline
    :\%s/.*/\textbackslash{}l\& &	Mettre la 1ère lettre de chaque ligne en minuscule\\ \hline
\end{tabular}\\
:set ignorecase Ignore case in searches\\
:set smartcase Ignore case in searches excepted if an uppercase letter is used

\begin{tabular}{|p{4cm}| l| }\hline
	u & Annulation \\ \hline
	U & Annulation sur la ligne actuelle \\ \hline
	CTRL-r & Restauration \\ \hline
    .  & Répéter la dernière commande \\ \hline
\end{tabular}\\

\subsection{Substitution}
 :s/regex/Substitute/flags
flag g (global) replace all matches
c ask confirmation
i insensitive case
number , replace the number th occurrence
no flag replace only the first occurrence

:s/foo/bar/g all occurence on the current line.
:$s/foo/bar/g the last line.
:\%s/foo/bar/g Change each 'foo' to 'bar' in all lines.
:5,12s/foo/bar/g Change each 'foo' to 'bar' for all lines from line 5 to line 12 inclusive.
:'a,'bs/foo/bar/g Change each 'foo' to 'bar' for all lines from mark a to mark b inclusive.
:.,\$s/foo/bar/g Change each 'foo' to 'bar' for all lines from the current line (.) to the last line ($) inclusive.
:.,+2s/foo/bar/g Change each 'foo' to 'bar' for the current line (.) and the two next lines (+2).
:g/\^baz/s/foo/bar/g Change each 'foo' to 'bar' in each line starting with 'baz'.
if no 'g' at the end , replace only the first or the 'n'th occurence
:g/regex1/s/regex2/string/g Replace all lines matching a regular expression from the set that matches another regular expression
For example, toreplace all occurrences of York with Orleans on every linethat matches New York, type :g/New York/s/York/Orleans/g
:\%s/\^/hello/g 	Replace the begining of each line by hello
:\%s/\$/Harry/g 	Replace the end of each line by Harry
:\%s/onward/forward/gi 	Replace onward by forward, case unsensitive
:\%s/ *\$//g 	Delete all white spaces
:g/string/d 	Delete all lines containing string
:v/string/d 	Delete all lines containing which didn t contain string
:%s#<[^>]\+>##g 	Delete HTML tags but keeps text
:%s/^\(.*\)\n\1$/\1/ 	Delete lines which appears twice


\section{Ouvrir, sauvegarder et quitter}
\subsection{Ouvrir depuis la ligne de commande}
\begin{tabular}{|p{4cm}| l| }
\hline
vim -R  fichier & Ouvre le fichier en lecture seule \\ \hline
vim +143 fichier & Ouvre le fichier et positionne le curseur à la ligne 143 \\ \hline
vim +/recherche fichier & Ouvre le fichier et positionne le curseur à la première correspondance du terme recherché depuis le début du fichier   \\ \hline
vim +?recherche fichier & Ouvre le fichier et positionne le curseur à la première correspondance du terme recherché depuis la fin du fichier \\ \hline
vim + fichier & Ouvre le fichier et positionne le curseur à la fin du fichier  \\ \hline
\end{tabular}\\


\subsection{Depuis Vim}
\begin{tabular}{|p{3cm}| l| }
\hline
ZZ or :wq or :x & Sauvegarder le fichier et quitter\\ \hline
ZQ or :q! & Quitter sans sauver\\ \hline
:w [nomfichier]& Sauvegarde dans nomfichier \\ \hline
:n,mw foo & Sauvegarde les ligne n à m dans 'foo' \\ \hline
:w \textgreater \textgreater foo & Ajoute le fichier courant au fichier ' foo'\\ \hline
:e ou :edit & Ouvrir un fichier (éditer) \\ \hline
:saveas foo & Pareil que :w foo \\ \hline
:wq! & Sauvegarde lorsqu'un fichier a été ouvert en lecture seule (vim -R file) \\ \hline
gf & Ouvre le fichier dont le nom est le mot sous le curseur\\ \hline
:Sex & Ouvrir un exploreur de fichier dans un nouveau split\\ \hline
:Ex & Ouvrir un exploreur de fichier\\ \hline
:browse e & Ouvre un exploreur de fichier graphique \\ \hline
:e! ou :edit! & Remet la dernière version sauvegarder du fichier \\ \hline
\end{tabular}\\

:ls List buffers
:cd .. Move to parent directory
:args List files
:args *.php Open file list
:grep expression *.php Returns a list of .php files contening expression
gf Open file name under cursor
\subsection{Les onglets}
vim -p file1 file2 ouvrir chaque fichier dans un onglet
:tabe ou :tabedit {file} ouvrir un fichier (éditer) dans un nouvel onglet
:tabnew ouvrir un nouvel onglet vide
:tabfind {file} open a new tab with filename given, searching the 'path' to find it
:tabclose close current tab
:tabclose {i} close i-th tab
:tabonly close all other tabs (show only the current tab)
:n Go to next file
:p Go to previous file

\section{Macro}
\begin{tabular}{|p{3cm}| l| }\hline
qa & Enregistre dans le registre a \\ \hline
@a & Rejoue la macro enregistrée dans le registre a \\ \hline
@@ & Rejoue la dernière macro éxécutée \\ \hline
10@a & Rejoue 10 fois la macro du registre a \\ \hline
\end{tabular}\\


\noindent
Exemple : Pour créer une séquence de nombre\\
\noindent
Ecrire 1. sur la première ligne, sortir du mode insertion avec <Esc> puis taper qaYp<CTRL-a>q\\
Explication :
\begin{itemize}
\renewcommand{\labelitemi}{$\bullet$}
\item qa : début de l'enregistrement.
\item Yp : copier cette ligne.
\item <CTRL-a> : incrémente le nombre.
\item q : arrête d'enregistrer.
    \end{itemize}
    Écrivez 100@@ ou 100@a pour créer une liste de nombre croissants jusqu'à 102.
\section{Marqueurs}
\begin{tabular}{|p{3cm}| l| }
\hline
mx & Add a marker to the character x\\ \hline
`x & Jump to marker x\\ \hline
'x & Jump to the first non-blank character on the line with marker\\ \hline
\end{tabular}\\

\section{Complétion}
Les commandes suivantes se font en mode insertion\\


\begin{tabular}{|p{3cm}| l| }
    \hline
CTRL-p & Recherche pour completion du mot dans le texte avant le curseur \\ \hline
CTRL-n & Recherche pour completion dans mot le texte après le curseur \\ \hline
CTRL-x CTRL-l &	Recherche pour completion de la ligne \\ \hline
\end{tabular}\\


\section{Les splits}
\subsection{Ouverture}
\begin{tabular}{|p{4cm}| l| }\hline
\multicolumn{2}{|l|}{\textbf{Lors du lancement de vim }} \\ \hline
    vim -o foo bar & Ouvre foo et bar en split\\ \hline
    vim -O foo bar & Ouvre foo et bar en vsplit\\ \hline
\multicolumn{2}{|l|}{\textbf{Depuis vim}}\\ \hline
:sp tux ou :split tux & Ouvrir le fichier tux dans un split horizontal \\ \hline
    :vsp tux ou :vsplit tux & Ouvrir le fichier tux dans un split vertical \\ \hline
    :sview tux & Pareil que :sp mais en lecture seule \\ \hline
\multicolumn{2}{|l|}{\textbf{Ouvrir de nouveaux splits }} \\ \hline
CTRL-w s & Partager l'écran en deux horizontalement \\ \hline
    CTRL-w v & Partager l'écran en deux verticalement \\ \hline
\end{tabular}\\

\subsection{Déplacement}

\begin{tabular}{|p{3cm}| l| }\hline
CTRL-w j & sélectionne le split d'en bas\\ \hline
CTRL-w k & sélectionne le split d'en haut\\ \hline
CTRL-w h & sélectionne le vsplit de gauche\\ \hline
CTRL-w l & sélectionne le vsplit de droite\\ \hline
\end{tabular}\\

\noindent
 CTRL-w fleche fonctionne aussi pour se déplacer. Si vous n'avez que deux splits, CTRL-w w vous permettra de basculer de l'un à l'autre.
\subsection{Arrangement}

\begin{tabular}{|p{3cm}| l| }\hline
\multicolumn{2}{|l|}{\textbf{Splits horizontals}} \\ \hline
    CTRL-w + & Agrandir le split actif d'une ligne\\ \hline
    CTRL-w - & Réduirele split actif d'une ligne\\ \hline
    CTRL-w \_ & Maximise la taille du split\\ \hline
\multicolumn{2}{|l|}{\textbf{Splits vertivals}}\\ \hline
    CTRL-w > & Agrandir le split actif d'une colonne \\ \hline
    CTRL-w < & Réduire le split actif d'une colonne \\ \hline
    CTRL-w | & Maximise la taille du split\\ \hline
\end{tabular}\\[1.5em]


\begin{tabular}{|p{3cm}| l| }\hline
    CTRL-w = & Égalise la taille des splits \\ \hline
    CTRL-r r & Échange la position des splits\\
    & Fonctionne aussi avec "R" majuscule pour échanger en sens inverse \\ \hline
    CTRL-w q & Fermer le split courant \\ \hline
    CTRL-w o & Fermer tous les splits saut le split courant \\ \hline
\end{tabular}\\


\section{Indentation}
\begin{tabular}{|p{3cm}| l| }
\hline
< & indenter à gauche\\ \hline
> & indenter à droite\\ \hline
== & autoindentation d'une ligne\\ \hline
= & auto-indentation de la sélection\\ \hline
V{deplacement}= & pour réindenter la sélection\\ \hline
=i\{ & pour réindenter l'intérieur du bloc d'accolades courant\\ \hline
=a\{, & les lignes avec les accolades aussi\\ \hline
=ap & pour réindenter le paragraphe courant\\ \hline
=\% & (re)indenter le bloc des accolades \{ ... \} \\ \hline
G=gg   & auto (re)indenter le fichier entier \\ \hline
\end{tabular}\\

\section{Insertion de fichier}
On va utiliser la commande \verb?:read? pour insérer dans notre fichier un autre fichier ou le résultat d'une commande\\


\begin{tabular}{|p{3cm}| l| }
\hline
:r foo.txt & Insérer le fichier foo.txt sous le curseur \\ \hline
:0r foo.txt & Insérer le fichier foo.txt avant la première ligne \\ \hline
:r !ls & Insérer le résultat de la commande \verb?ls?  sous le curseur \\ \hline
:nr !ls & Insérer le résultat de la commande \verb?ls? à la ligne n \\ \hline
:'ar !ls & Insérer le résultat de la commande \verb?ls? à la ligne avec le marqueur a \\ \hline
:\$r !pwd & Insérer le résultat de la commande \verb?ls? à la fin du fichier  \\ \hline
:\%!cmd & Replacer tout le fichier par le résultat de la commande \verb?cmd? \\ \hline
:n,m!cmd & Replacer la ligne n à m par le résultat de la commande \verb?cmd? \\ \hline
:n!cmd & Replacer la ligne n par le résultat de la commande \verb?cmd? \\ \hline
:.!cmd & Replacer la ligne actuelle par le résultat de la commande \verb?cmd?\\ \hline
:'a !cmd & Replacer la ligne avec le marqueur a par le résultat de la commande \verb?cmd? \\ \hline
\end{tabular}\\


\section{Divers}
\begin{tabular}{|p{4cm}| l| }
\hline
:ab test test@foo.info & l'abbréviaton test sera remplacé par test@foo.info \\ \hline
:!cmd ou !!cmd & Lancer la commande externe \verb?cmd?  \\ \hline
    CTRL-a &sur un nombre Ajoute 1 au nombre \\ \hline
    CTRL-x &sur un nombre Enleve 1 au nombre \\ \hline
    ggg?G   & rot13 le fichier entier \\ \hline
    ggVGg? 	& Changer le texte to Rot13 \\ \hline
    :w !sudo tee \% &  modifier un fichier système lorsqu'il a été ouvert sans être root \\ \hline
\end{tabular}\\

\end{document}
