\documentclass{article}
\usepackage[utf8]{inputenc} 
\usepackage[T1]{fontenc}
\usepackage{lmodern}
\usepackage{graphicx}
\usepackage[french]{babel}
\usepackage[bookmarks=true]{hyperref}
\usepackage{geometry}
\geometry{hmargin=2.5cm,vmargin=2cm}

\title{Raccourcis Vim}
\author{Adrian Poiget}

\begin{document}
\maketitle
\begin{abstract}
Ce document contient une liste non exhaustive des différents raccourcis sur VIM.
J'espère qu'elle pourra vous être utile.\\
Les raccourcis seront représenté comme dans le tableau suivant~: 
\begin{tabular}{|p{3cm}| l|  }
\hline
Raccourcis & Action\\ \hline
\end{tabular}\\
Pour les raccourcis, la casse est importante : les lettres en majuscule sont l'équivalent de la combinaison Maj-lettre. Par exemple, V correspond à Maj-v et v correspond uniquement à la touche v.
\end{abstract}

\tableofcontents
\newpage

\section{Se déplacer}
\subsection{Déplacement de base}
\begin{tabular}{|p{3cm}| l|  }
	\hline
	h & Aller à gauche \\ \hline
	j & Aller en bas \\ \hline
	k & Aller en haut \\ \hline
	l & Aller à droite \\ \hline
\end{tabular}
ou alors on peut aussi utiliser les flèches.

\subsection{Déplacement par mots}
\begin{tabular}{|p{3cm}| l|  }
	\hline
	w & Aller au début du mot suivant\\ \hline
	W & Aller au début du MOT suivant\\ \hline
	e & Aller à la fin du mot suivant\\ \hline
	E & Aller à la fin du MOT suivant\\ \hline
	b & Aller au début du mot précédent\\ \hline
	B & Aller au début du MOT précédent\\ \hline
    ge & Aller à la fin du mot précédent\\ \hline
	gE & Aller à la fin du MOT précédent\\ \hline
\end{tabular}\\

mot vs MOT : mot est une chaine de lettres, chiffres et underscores, MOT est une chaine de caractères sans espace\\

\subsection{Déplacement dans le fichier}
\begin{tabular}{|p{3cm}| l|  }
	\hline
	0 (zero) & Aller au début de la ligne \\ \hline
	\$ & Aller à la fin de la ligne \\ \hline
	\^~(accent circonflexe) & Aller au premier caratère non vide de la ligne \\ \hline
	g\_  & Aller au dernier  caratère non vide de la ligne \\ \hline
	+ & Aller au premier caractère non blanc de la ligne suivante \\ \hline
	- & Aller au premier caractère non blanc de la ligne précédente \\ \hline
	"Entrée" & Aller au premier caractère non blanc de la ligne suivante\\ \hline
	:0 & Aller au début du fichier (ligne 0) \\ \hline
	gg & Aller au début du fichier (ligne 0) \\ \hline
	1G & Aller au début du fichier (ligne 0) \\ \hline
	:\$ & Aller à la fin du fichier\\ \hline 
	G  & Aller à la fin du fichier\\ \hline
	CTRL-d & Aller à la fin du fichier\\ \hline
	:n  & Aller à la ligne n\\ \hline
	nG  & Aller à la ligne n\\ \hline
	ngg & Aller à la ligne n\\ \hline
\end{tabular}\\[0.5cm]



\begin{tabular}{|p{3cm}| l|  }
	\hline
  50\% & Aller à 50\% du fichier.(milieu du fichier) \\ \hline
  100l & Aller au 100ème caractere depuis la position acutelle \\ \hline
  100  (100 suivi d'un espace) & Aller au 100ème caractere depuis la position acutelle \\ \hline
  25| & Aller au 25ème caractère de la ligne en cours\\ \hline
  :goto 25 & Aller au 25ème caractère depuis le début du fichier\\ \hline
\end{tabular}\\


\begin{tabular}{|p{3cm}| l|  }
	\hline
	fx & Déplace le curseur à la prochaine occurence du caractère x \\ \hline
	Fx & Déplace le curseur à la précédente occurence du caractère x \\ \hline
	tx & Déplace le curseur juste avant la prochaine occurence du caractère x \\ \hline
	Tx & Déplace le curseur juste avant la précédente occurence du caractère x \\ \hline
\end{tabular}\\

Pour ce déplacer à l'occurence suivante, il suffit d'appuyer sur ; et pour aller à l'occurrence  précédente c'est le raccourcis ,
    
\begin{tabular}{l l}
Exemple&	4fx  recherche la 4ième occurrence de x.\\
	&dt"  supprime tout jusqu'au prochain ".\\
	&2dfx  supprime tout jusqu'à la 2eme prochaine occurence de  x.\\
\end{tabular}\\

\begin{tabular}{|p{3cm}| l|  }
	\hline
	CTRL-n & Descendre le curseur d'une ligne \\ \hline
	CTRL-p & Monter le curseur d'une ligne\\ \hline
	CTRL-f & Descendre le curseur d'une page\\ \hline
	CTRL-b & Monter le curseur d'une page\\ \hline
	CTRL-d & Descendre le curseur d'une demi-page\\ \hline
	CTRL-u & Monter le curseur d'une demi-page\\ \hline
	CTRL-e & Descendre l'affichage d'une ligne\\ \hline
	CTRL-y & Monter l'affichage d'une ligne\\ \hline
	L & Aller au bas de l'écran\\ \hline
	xL  & Placer le curseur à x lignes du bas de l'écran \\ \hline
	H & Aller en haut de l'écran \\ \hline
	xH & Placer le curseur à x lignes du haut de l'écran \\ \hline
	M & Aller au milieu de l'écran\\ \hline
\end{tabular}\\
\subsection{Déplacement sur une ligne de "plusieurs lignes" visuellement }
\begin{tabular}{|p{3cm}| l|  }
	\hline
gj & Descendre d'une "ligne"\\ \hline
gk & Monter d'une "ligne"\\ \hline
g0 & Aller au début de la "ligne"\\ \hline
gm & Aller au milieu de la "ligne"\\ \hline
g\$ & Aller à la fin de la "ligne"\\ \hline
g\^~ & Aller au premier caratère non vide  de la ligne\\ \hline
\end{tabular}\\

\subsection{Sauts}
Lorsquen vous vous déplacer dans le fichier grâce aux raccourcis précédents, vous pouvez revenir à la position d'avant avec les raccourcis suivavnts~: \\


\begin{tabular}{|p{3cm}| l|  }
	\hline
CTRL-o & Revenir à l'emplacement précédent\\ \hline
CTRL-i ou tab & Aller à l'emplacement suivant\\ \hline
5CTRL-o & Sauter au 5ème emplacement précédent\\ \hline
5CTRL-i & Sauter au 5ème emplacement suivant\\ \hline
:jumps & Afficher les postions de jumps\\ \hline
\end{tabular}\\

\section{Rechercher}  
\begin{tabular}{|p{3cm}| l|  }
	\hline
/chaine & Recherche le mot vers le bas\\ \hline
?chaine & Recherche le mot vers le haut\\ \hline
\end{tabular}\\

\noindent
Aller à la prochaine occurence de la chaine dans la même direction en tapant n.\\
Aller à la prochaine occurence de la chaine dans la même direction en tapant N

\section{Information}
Voici maintenant quelques commandes pour obtenir des informations sur la ligne ou le caractère\\


\begin{tabular}{|p{3cm}| l|  }
	\hline
	:= & Affiche le nombre de ligne\\ \hline
	CTRL-g & Affiche des informations sur le ligne en cours\\ \hline
	g CTRL-g & Affiche des informations détaillées sur le ligne en cours\\ \hline
	:.=  & Affiche le numero de la ligne en cours\\ \hline
	:\#   & Affiche le numero de la ligne en cours\\ \hline
	:num & Affiche le numero de la ligne en cours\\ \hline
	:/foo/= & Affiche le numero de la prochaine ligne contenant foo\\ \hline
	ga & Affiche la valeur ASCII du caractère\\ \hline
\end{tabular}\\

\section{Insertion/Edition}
\subsection{Insertion}
\begin{tabular}{|p{3cm}| l|  }
	\hline
	i & Passer en mode insertion avant le curseur\\ \hline
	I & Passer en mode insertion au début de la ligne\\ \hline
	a & Passer en mode insertion après le curseur\\ \hline
	A & Passer en mode insertion à la fin de la ligne\\ \hline
	o & Ajouter une ligne en dessous et passer en mode insertion \\ \hline
	O & Ajouter une ligne au dessus et passer en mode insertion \\ \hline
\end{tabular}\\
\subsection{Edition}
\begin{tabular}{|p{3cm}| l|  }
	\hline
	r & Remplacer un caractère (5ra remplace les 5 caractère par a)\\ \hline
	R & Remplacer plusieurs caractères\\ \hline
	cw & Effacer le mot et passer en mode insertion (changement)\\ \hline
	cc ou S & Effacer la ligne entière et passer en mode insertion (changement)\\ \hline
	c\$ ou C & Effacer jusqu'à la fin de la ligne et passer en mode insertion (changement)\\ \hline
	cl ou s  & Effacer le caractère et passer en mode insertion \\ \hline
	xp & transposer deux caractères\\ \hline
	J & joindre 2 lignes (nJ pour joindre n lignes)\\ \hline
	gJ  &joindre 2 lignes sans l'espace\\ \hline
\end{tabular}\\

	select all lines ggVG or ggHG (gggHG ?)
	gf go to the file named by the word
	u undo
	U undo on the current line
	Ctrl-r redo
    
\subsection{Suppression}
\begin{tabular}{|p{3cm}| l|  }\hline
x ou dl & supprimer le caractère sous le curseur (4x en supprime 4)\\ \hline
X ou dh & supprimer le caractère placé avant le curseur\\ \hline
dd ou :.d & supprimer la ligne entière (3dd poursupprimer 3 lignes)\\ \hline
dw & supprimer la fin du mot (d4w delete 4 words)\\ \hline
de & supprimer la fin du mot mais pas l'espace de fin \\ \hline
db & supprimer le début du mot mais pas l'espace de fin\\ \hline
dge & supprimer le début du mot et l'espace de début\\ \hline
d\$ or D & supprimer jusqu'à la fin de la ligne\\ \hline
d0 & supprimer le début de la ligne\\ \hline
diw & supprimer le mot (excluant l'espace)\\ \hline
daw & supprimer le mot (incluant l'espace)\\ \hline
dG & supprimer la fin du fichier\\ \hline
dgg & supprimer le début du fichier\\ \hline
\end{tabular}\\


\subsection{Changer la casse}
\begin{tabular}{|p{3cm}| l|  }\hline
    $\sim$ & Changer la casse d'une lettre \\ \hline
	guu ou Vu& Toute la ligne est mise en minuscule\\ \hline
	gUU ou VU & Toute la ligne est mise en majuscule\\ \hline
	g$\sim$$\sim$ or g$\sim$g$\sim$ & Inverser la casse de la ligne\\ \hline
	vEU & Mettre le mot en majuscule \\ \hline
	vEu & Mettre le mot en minuscule\\ \hline
	vE$\sim$ & Inverser la casse du mot\\ \hline
	ggguG & Tout le fichier est mis en minuscule\\ \hline
	gggUG & Tout le fichier est mis en majuscule\\ \hline
	<selection>U & La sélection est mise en majuscule) \\ \hline
	<selection>u & La sélection est mise en minuscule \\ \hline
	g$\sim$motion & (It does not depend on tildeop) \\ \hline
	gUmotion & (All uppercase) \\ \hline
	gumotion & (All uppercase) \\ \hline
	gUw & Mettre le mot en majuscule\\ \hline
	guw & Mettre le mot en minuscule \\ \hline
	g$\sim$w & Inverser la casse du mot \\ \hline
    :\%s/\textbackslash{}\textless{}./\textbackslash{}u\&/g & Mettre la 1ère lettre de chaque mot en majuscule \\ \hline
    :\%s/\textbackslash{}\textless{}./\textbackslash{}l\&/g & Mettre la 1ère lettre de chaque mot en minuscule \\ \hline
    :\%s/.*/\textbackslash{}u\& &	Mettre la 1ère lettre de chaque ligne en majuscule\\ \hline
    :\%s/.*/\textbackslash{}l\& &	Mettre la 1ère lettre de chaque ligne en minuscule\\ \hline
\end{tabular}\\
	:set ignorecase 	Ignore case in searches\\
	:set smartcase 	Ignore case in searches excepted if an uppercase letter is used

\section{Ouvrir, sauvegarder et quitter}
\begin{tabular}{|p{3cm}| l|  }
	\hline
	ZZ or :wq or :x & Sauvegarder le fichier et quitter\\ \hline
	ZQ or :q! & Quitter sans sauver\\ \hline
	:w [nomfichier]& Sauvegarde dans nomfichier \\ \hline
	:n,mw foo & Sauvegarde les ligne n à m dans 'foo' \\ \hline
	:w \textgreater \textgreater  foo & Ajoute le fichier courant au fichier ' foo'\\ \hline
	:e ou :edit & Ouvrir un fichier (éditer) \\ \hline
	:saveas  foo & Pareil que :w foo \\ \hline
	:wq! & Sauvegarde lorsqu'un fichier a été ouvert en lecture seule (vim -R file)\\ \hline
	:Sex & Ouvrir un exploreur de fichier  dans un nouveau split\\ \hline
	:Ex & Ouvrir un exploreur de fichier\\ \hline
	:browse e 	& Ouvre un exploreur de fichier graphique\\ \hline
\end{tabular}\\
	
	:ls 	List buffers
	:cd .. 	Move to parent directory
	:args 	List files
	:args *.php 	Open file list
	:grep expression *.php 	Returns a list of .php files contening expression
	gf 	Open file name under cursor
\subsection{Les onglets}
	vim -p file1 file2 ouvrir chaque fichier dans un onglet
	:tabe  ou :tabedit {file}   ouvrir un fichier (éditer) dans un nouvel onglet
	:tabnew 		ouvrir un nouvel onglet vide
	:tabfind {file}   open a new tab with filename given, searching the 'path' to find it
	:tabclose         close current tab
	:tabclose {i}     close i-th tab
	:tabonly          close all other tabs (show only the current tab)
	:n	Go to next file
	:p	Go to previous file

\section{Marqueurs}
\begin{tabular}{|p{3cm}| l|  }
	\hline
mx & Add a marker to the character x\\ \hline
	`x & Jump to marker x\\ \hline
	'x & Jump to the first non-blank character on the line with marker\\ \hline
\end{tabular}\\

\section{Complétion}
Les commandes suivantes se font en mode insertion\\


\begin{tabular}{|p{3cm}| l|  }
    \hline
CTRL-P & Recherche pour completion du mot dans le texte avant le curseur \\ \hline
CTRL-N & Recherche pour completion dans mot le texte après le curseur \\ \hline 
CTRL+x CTRL+l &	Recherche pour completion de la ligne \\ \hline
\end{tabular}\\


\section{Les splits} 
\subsection{Ouverture}
\begin{tabular}{|p{3cm}| l|  }\hline
\multicolumn{2}{|l|}{\textbf{Lors du lancement de vim }} \\ \hline
    vim -o foo bar & Ouvre foo et bar en split\\ \hline
    vim -O foo bar & Ouvre foo et bar en vsplit\\ \hline
\multicolumn{2}{|l|}{\textbf{Depuis vim}}\\ \hline
	:sp tux ou :split tux & Ouvrir le fichier tux dans un split horizontal	\\ \hline
    :vsp tux ou :vsplit tux & Ouvrir le fichier tux dans un split vertical	\\ \hline
    :sview tux & Pareil que :sp mais en lecture seule \\ \hline
\multicolumn{2}{|l|}{\textbf{Ouvrir de nouveaux splits }} \\ \hline
	CTRL-w s & Partager l'écran en deux horizontalement	\\ \hline
    CTRL-w v & Partager l'écran en deux verticalement	\\ \hline
\end{tabular}\\

\subsection{Déplacement}

\begin{tabular}{|p{3cm}| l|  }\hline
CTRL-w j & sélectionne le split d'en bas\\ \hline
CTRL-w k & sélectionne le split d'en haut\\ \hline
CTRL-w h & sélectionne le vsplit de gauche\\ \hline
CTRL-w l & sélectionne le vsplit de droite\\ \hline
\end{tabular}\\

\noindent
 CTRL-w fleche fonctionne aussi pour se déplacer. Si vous n'avez que deux splits, CTRL-w w vous permettra de basculer de l'un à l'autre.
\subsection{Arrangement}

\begin{tabular}{|p{3cm}| l|  }\hline
\multicolumn{2}{|l|}{\textbf{Splits horizontals}} \\ \hline
    CTRL-w + & Agrandir le split actif d'une ligne\\ \hline
    CTRL-w - & Réduirele split actif d'une ligne\\ \hline
    CTRL-w \_ & Maximise la taille du split\\ \hline
\multicolumn{2}{|l|}{\textbf{Splits vertivals}}\\ \hline
    CTRL-w > & Agrandir le split actif d'une colonne \\ \hline
    CTRL-w < & Réduire le split actif d'une colonne \\ \hline
    CTRL-w | & Maximise la taille du split\\ \hline
\end{tabular}\\[1.5em]


\begin{tabular}{|p{3cm}| l|  }\hline
    CTRL-w = & Égalise la taille des splits \\ \hline
    CTRL-r r & Échange la position des splits\\
    & Fonctionne aussi avec "R" majuscule pour échanger en sens inverse \\ \hline
    CTRL-w q & Fermer le split courant \\ \hline
    CTRL-w o & Fermer tous les splits saut le split courant \\ \hline
\end{tabular}\\


\section{Indentation}
\begin{tabular}{|p{3cm}| l|  }
	\hline
	== & autoindentation d'une ligne\\ \hline
	= & auto-indentation d'un bloc\\ \hline
	V{deplacement}= & pour réindenter la sélection\\ \hline
	=i\{ & pour réindenter l'intérieur du bloc d'accolades courant\\ \hline
	=a\{, & les lignes avec les accolades aussi\\ \hline
	=ap & pour réindenter le paragraphe courant\\ \hline
	< & indenter à gauche\\ \hline
	> & indenter à droite\\ \hline
\end{tabular}\\

\section{Divers}
\begin{tabular}{|p{4cm}| l|  }
	\hline
	:ab test test@foo.info 	& l'abbréviaton test sera remplacé par test@foo.info \\ \hline
    CTRL-a &sur un nombre Ajoute 1 au nombre\\ \hline
    CTRL-x &sur un nombre Enleve 1 au nombre\\ \hline
\end{tabular}\\

\end{document}
